%!TEX root = ../VorlageBA.tex
\chapter{Kapitel 2}

\section{Section}
\label{sec:section}

Beispiel für eine Grafik:
\begin{figure}[!ht]
	\centering
		%[natürliche Breite in Pixeln, natürliche Höhe in Pixeln, Abhängigkeit von der Textbreite]
		\includegraphics[natwidth=1200pt, natheight=349pt, width=1.0\textwidth]{images/showcase.pdf}
	\caption{Bildunterschrift}
	\label{fig:showcase}
\end{figure}

\section{Section}
Beispiel für ein Zitat:
\begin{quotation}
	\textit{\enquote{A persona is a rich picture of an imaginary person who represents your core user group.}} \cite{Dix04}
\end{quotation}

\section{Section}
\nomenclature{Usability}{Das Ausmaß, in dem ein Produkt durch bestimmte Benutzer in einem Nutzungskontext benutzt werden kann, um ein Ziel effektiv, effizient und zufriedenstellend zu erreichen.\vspace{4mm}}

\section{Section}

\section{Section}

\section{Section}


%======================================================================
%======================================================================
% Eintrag ins Glossar
\nomenclature{Usability}{Das Ausmaß, in dem ein Produkt durch bestimmte Benutzer in einem Nutzungskontext benutzt werden kann, um ein Ziel effektiv, effizient und zufriedenstellend zu erreichen.\vspace{4mm}}